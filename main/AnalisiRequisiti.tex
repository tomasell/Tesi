\section{Analisi dei requisiti}
Durante il periodo di stage che ho sostenuto presso SAIV S.p.A., in
collaborazione con il tutor aziendale dott. Lovato Giovanni e il collega Rigoni
Giulio , è stata sviluppata un'applicazione web per monitorare il livello di
\ac{CS}.

Questa applicazione è stata creata a seguito di una reale richiesta di una
commessa da parte di un'azienda. 
L'Ente utilizza già un questionario cartaceo aderente al modello
\emph{SERVQUAL}, ma richiedeva uno strumento intuitivo da affiancare al
questionario stesso così da ottenere dei \emph{feedback} anche dai clienti che
ritenevano troppo impegnativa la compilazione del test. 
Un'applicazione quindi finalizzata a coinvolgere
quanti più utenti possibili, ad avere una maggiore quantità di dati ed
un'analisi immediata e automatizzata. Il nuovo modello di acquisizione dei dati non è stato
specificato, e non ci erano stati imposti degli standard da seguire, validi
per il monitoraggio del \emph{customer satisfaction}. Al committente premeva
presentare ai clienti uno strumento di rapido e facile utilizzo. 
La ditta è presente sul territorio nazionale con più sedi dislocate, ognuna
delle quali dovrà essere fornita del nuovo strumento ad eccezione della sede
centrale, l'unica autorizzata all'analisi e visualizzazione dei dati. Le informazioni
raccolte verranno differenziate per filiale e periodo di osservazione. L'analisi
dei voti dovrà essere consultata su grafici di diverso tipo.
Inoltre l'applicazione da presentare ai clienti deve essere utilizzata su
dispositivi touch screen o totem multimediali posti all'interno delle sedi.
Infine richiedevano che fosse possibile inserire pubblicità a scopi commerciali,
video ed informazioni di servizi e promozioni che l'azienda mette a
disposizione.
\\\\ 
Per diritti di privacy aziendale sostituirò il nome del committente con
una banca fittizia, la Banca d'Europa, e ho adattato l'applicazione
finalizzandola a monitorare i servizi offerti dalle varie sedi venete.