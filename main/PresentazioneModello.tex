\newpage
\section{Presentazione del modello}
La nostra applicazione è certamente più veloce ed intuitiva rispetto al modello
ServQual. Si tratta di una modalità che
permette di avere un feedback generale e veloce, ma non completo. Per essere un modello
valido sarebbe necessario elaborarlo ulteriormente, aggiungendo, per esempio,
domande riguardanti le aspettative dei clienti rispetto i servizi offerti ed una
scala più ampia come valore delle risposte.
Altro punto di forza dell'applicazione è l'automatizzazione. A
differenza del questionario cartaceo in cui poi risulta necessario registrare
manualmente i dati, elaborarli ed archiviarli, l'applicazione permette queste
operazioni in tempi veloci e contemporaneamente all'espressione del voto da
parte del cliente, in modalità automatica.
È necessario, però, evidenziare che il nostro modello non è esaustivo, ma è
un'applicazione da affiancare ad altri strumenti. 
\\\\
L'utente ha la possibilità di esprimere la propria valutazione scegliendo tra le
possibilità: non del tutto soddisfatto, mediamente soddisfatto, soddisfatto e
molto soddisfatto. Ad ogni alternativa abbiamo conferito un valore: a ``non del tutto
soddisfatto'' il valore 1, a ``mediamente soddisfatto'' il valore 3, a
``soddisfatto'' il valore 5 e a ``molto soddisfatto'' il valore 7. La scelta dei
valori è stata fatta per permettere, in un tempo successivo, l'inserimento di
altre valutazioni intermedie (2,4,6) che faciliterebbero una visione più
completa e andrebbero a rispettare la scala di valutazione prevista da modello
ServQual.

All'utente sarà permesso di esprimere il proprio voto in modo anonimo,
intuitivo e veloce con la possibilità di aggiungere un commento che verrà
registrato dal sistema. L'amministratore, invece, potrà analizzare i dati, anche
in tempo reale, visualizzati su grafici. 

Ad una prima analisi della commessa con
il team, è stata presa la decisione di realizzare un'applicazione con una
sezione \emph{front end}, ottimizzata per essere eseguita su dispositivi
touchscreen e una sezione \emph{back end} accessibile solo dagli amministratori.
Nel momento in cui ci siamo chiesti quali dati sarebbe stato utile salvare,
abbiamo deciso di registrare nel sistema il commento, il nominativo e l'email
dell'utente (dati presentati come facoltativi, con il modello per la privacy e
l'utilizzazione dei dati), la data di votazione, la
sede, il tipo di servizio e lo score. Il commento personale
permetterà all'ente di avere un riscontro più completo rispetto alla singola
risposta di scelta multipla e molto utile al fine di una verifica dei servizi
proposti, il nome e l'indirizzo di posta elettronica consentiranno al
committente di contattare la persona che lo ha richiesto e la data di votazione
può facilitare l'analisi dell'affluenza al servizio.
