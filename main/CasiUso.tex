\newpage
\section{Casi d'uso}
Nell'applicazione qui presentata ci sono due casi d'uso: uno riservato agli
utenti~(Figura~\ref{fig:usecase-voting}), l'altro agli
amministratori~(Figura~\ref{fig::usecase-monitoring}).
\subsection{Inserimento della votazione \emph{front end}}
\begin{itemize}
  \item \textbf{Voto in uscita} : in uscita dalla sede il cliente
  troverà un totem multimediale touch screen con cui potrà esprimere il
  proprio livello di soddisfazione dopo aver usufruito dei servizi offerti. La
  schermata di introduzione è composta da un messaggio che spiega
  all'utilizzatore la finalità per la quale viene richiesto di lasciare un
  \emph{feedback}, un video introduttivo realizzato dell'azienda a scopi
  commerciali e la scelta dell'operazione effettuata: servizio ricevuto presso
  lo sportello oppure una consulenza sulle operazioni finanziarie.
  \item \textbf{Inserimento commento} : l'utente può lasciare un proprio
  commento selezionando l'area con la voce ``Lasciaci un commento''. Comparirà
  una finestra di dialogo che permetterà l'inserimento del testo , il nome e
  cognome e l'indirizzo email nel caso in cui il cliente abbia piacere di essere
  contattato.
  Per registrare il commento bisogna accettare le condizioni sulla
  privacy.
  \item \textbf{Registrazione del voto} : il cliente esprime il proprio livello
  di soddisfazione tramite l’interfaccia selezionando uno dei 4 bottoni.
  \item \textbf{Feedback all'utente} : Terminato il processo di registrazione
  una finestra di dialogo certifica all'utente che il processo di voto si è
  concluso in modo corretto.
\end{itemize}

\begin{figure}[!h]
  \centering
  \begin{tikzpicture}
    \begin{umlsystem}[x=6]{Scorecard}
      \umlusecase[name=sportello]{Operazione allo sportello}
      \umlusecase[name=comment,y=-1.7]{Inserimento commento}
      \umlusecase[name=voting,y=-3.3]{Espressione del voto}
      \umlusecase[name=store,y=-5]{Registrazione del voto}
    \end{umlsystem}
    \umlactor{Cliente}
    \umlactor[x=12,y=-5]{Sistema}
    \umlassoc{Cliente}{sportello}
    \umlassoc{Sistema}{store}
    \umlextend{comment}{sportello}
    \umlassoc{comment}{voting}
    \umlinclude{store}{voting}
  \end{tikzpicture}
  \caption{Caso d'uso votazione}
  \label{fig:usecase-voting}
\end{figure}

\subsection{Consultazione dei voti \emph{back end}}
L'amministratore potrà consultare l'andamento del livello di \acf{CS} via
browser web.
L’interfaccia di riepilogo dei voti presenterà in modo grafico i dati raccolti:
\begin{itemize}
  \item l'amministratore accede all'interfaccia web del sistema.
  \item l'amministratore sceglie la/e sedi da monitorare, l'intervallo di
  tempo e la tipologia di grafico. Il grafico generato conterrà in ascissa
  l'intervallo di tempo scelto ed in ordinata il valore del \ac{CS}. A seconda
  del periodo da monitare l'intevallo di campionamento è espresso in ore per la
  modalità ``giornaliera'', giorni per ``mensile'' e mesi per
  ``annuale''.
  \item la selezione del valore ``automatico'' aggiornerà il grafico ogni 5
  secondi aggiornando i valori qual'ora venisse registrata una nuova votazione.
\end{itemize}

\begin{figure}[!h]
  \centering
  \begin{tikzpicture}
    \begin{umlsystem}[x=6]{Scorecard}
      \umlusecase[name=monitoring]{Accesso a interfaccia web}
      \umlusecase[name=graph,y=-2]{Generazione dei grafici}
    \end{umlsystem}
    \umlactor{Amministratore}
    \umlactor[x=12,y=-2]{Sistema}
    \umlassoc{Amministratore}{monitoring}
    \umlassoc{Sistema}{graph}
    \umlinclude{graph}{monitoring}
  \end{tikzpicture}
  \caption{Caso d'uso consultazione dei voti}
  \label{fig::usecase-monitoring}
\end{figure}
