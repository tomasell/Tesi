\newpage
\section{Scelte progettuali}
Dopo aver raccolto le richieste del committente, il team di sviluppo le ha
analizzate per decidere come implementare l'applicazione ed offrire così una
proposta di risoluzione al cliente. Sono stati considerati alcuni dettagli
implementativi, i costi e il tempo di sviluppo necessario per la realizzazione.
\\\\
Tenendo conto che l'applicazione doveva essere indipendente dalla piattaforma
usata, in quanto doveva essere ospitata su dipositivi touch screen che possono
essere sostituiti con prodotti concorrenziali o totem multimediali,la scelta è
stata quella di sviluppare una \emph{web application}, residente su un server,
accessibile tramite browser web. I linguaggi, quindi, con cui abbiamo
sviluppato il sistema sono \ac{HTML}, \ac{CSS} e
\ac{JS}. Il sistema di votazione è accessibile da tutte le sedi ed è
composto da una sezione specifica per le offerte e pubblicità aziendali e da
un'altra per l'inserimento del voto in modo anonimo. L'applicazione dedicata
agli amministratori, distinta da quella di votazione , è finalizzata
all'analisi dei dati e alla generazione dei grafici.
\\\\
Essendo il sistema ancora in fase di sviluppo e dovendo ancora il committente
analizzare e valutare la soluzione che abbiamo proposto, il modello che useremo
potrà essere cambiato e potrà acquisire nuove proprietà. Per lo storage dei
dati, dunque, avevamo il bisogno di usare un \ac{DBMS} non relazionale, basato
sui documenti e non vincolato ad una struttura tabellare prefissata del modello
rappresentato. L'utilizzo dei documenti per il salvataggio dei dati lascia maggiori
libertà di modifica degli attributi del modello. Avendo deciso di sviluppare
\emph{web application}, la scelta del database per lo storage dei dati è
ricaduta su CouchDB. Questo database è interrogabile tramite richieste
\ac{HTTP} e permette, inoltre, la sincronizzazione automatica tra database
ospitati su host diversi. In questo modo è possibile sincronizzare i dati presenti nei database
delle filiali con la sede centrale.

