\newpage
\section{Scelte progettuali}
Raccolte le richieste del committente il team di sviluppo le ha analizzate per
decidere come implementare l'applicazione e offrire una proposta
di risoluzione al cliente con alcuni dettagli implementativi, costi e
tempo di sviluppo.
\\\\
Tenendo conto che l'applicazione dovrà essere indipendente dalla piattaforma
usata in quanto dovrà essere ospitata su dipositivi touch screen che possono
essere sostituiti con prodotti concorrenziali o totem multimediali,la scelta è
stata quella di sviluppare una \emph{web application}, residente su un server,
accessibile tramite browser web. I linguaggi, quindi, con cui verrà sviluppato
il sistema sono HyperText Markup Language (HTML), Cascading Style Sheets (CSS) e
JavaScript (JS). Il sistema di votazione sarà accessibile dalle sedi e sarà
composto da una sezione dedicata alle offerte e pubblicità aziendali e una
sezione per l'inserimento del voto in modo anonimo. La sezione dedicata agli
amministratori, sarà distinto da quello di votazione , e finalizzato all'analisi
e generazione dei grafici.
\\\\
Essendo il sistema ancora in fase di sviluppo e la soluzione che proporremo
dovrà essere analizzata e valutata dal committente, il modello che
useremo verrà cambiato ed acquisirà nuove proprietà. Per lo storage dei
dati,dunque,avevamo il bisogno di usare un DBMS non relazionale, basato
sui documenti e non vincolato a una struttura tabellare del modello
rappresentato prefissata. L'utilizzo dei documenti per il salvataggio dei dati
lascia maggiori libertà di modifica degli attributi del modello. Avendo
deciso di sviluppare \emph{web application}, la scelta del database per lo
storage dei dati è ricaduta su CouchDB. Questo database inoltre è interrogabile
tramite richieste HTTP e permette la sincronizzazione automatica tra database
ospitati su host diversi. In questo modo è possibile sincronizzare i dati
presenti nei database delle filiali con la sede centrale.

