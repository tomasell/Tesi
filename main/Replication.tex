\newpage
\subsection{Replication} 
CouchDB consente si sincronizzare database situati su differenti host
permettendo quindi agli utenti di accedere a risorse situate su host diversi in
modo trasparente.
La sincronizzazione è possibile perchè CouchDB è un database
NOSQL e quindi ogni base di dati è una collezione di documenti indipendenti tra
di loro, cioè che documenti diversi possono rappresentare oggetti
diversi.
La sincronizzazione è unilaterale, bisogna indicare un server che conterrà
il database originale (\emph{source}) e uno con funzione di \emph{target}.
Un database viene definito ``locale'' quando si trova sulla stessa istanza del
server al quale viene inviata la richiesta di replication.
Tutti gli altri casi i database sono definiti ``remoti''.
Per sincronizzare un database si effettua una richiesta POST, inviando nel body
un documento JSON contenente i database di source e destination:

\begin{lstlisting}[language=json]
//file replication.json:
{
  source:"database",
  target:"http://example.org/database"
}
\end{lstlisting}
 
 
\emph{curl -X POST http://atcustomersat:5984/\_replicate -H ``Content-Type:}
\emph{application/json'' --data @replication.json}
\\\\
Quando si utilizza la sincronizzazione, CouchDB confronta i due database e
verifica quali documenti presenti nel source differiscono da quelli presenti nel
target. I documenti alloccati nel target verranno modificati , cancellati o
creati a seconda dello stato di quelli presenti nell'origine. Il numero di
revisione del documento viene modificato, ed è possibile risalire alle modifiche
effettuate al momento della sincronizzazione. Per realizzare una
sincronizzazione bilaterale basta inviare due richieste di replication
invertendo target e source.

Aggiungendo il parametro continuous e settandolo a true, è possibile continuare
a sincronizzare i database ogni volta che vengono aggiunti o modificati
documenti nel database di source. Nell'attuale versione di CouchDB ( v1.3.0) non
è possibile continuare la sincronizzazione dopo il riavvio del server, in quanto
i settaggi della replication vengono resettati.
