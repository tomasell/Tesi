\newpage
\subsection{API}
Il database è accessibile tramite richieste \acf{HTTP}, le cui proprietà sono
specificate nella \ac{RFC} 2616. Una richiesta
\ac{HTTP} è formata da una riga di richiesta (\emph{request line}) che contiene
il metodo, lo \ac{URI} e la versione del protocollo. La seconda componente della
richiesta è la sezione \emph{header} , la terza il \emph{body}. 

Attraverso i metodi messi a disposizione dal protollo,
è possibile memorizzare nuovi dati e creare nuove viste (\emph{Create}),
richiedere informazioni (\emph{Read}), modificare le informazioni memorizzate
all'interno dei documenti (\emph{Update}) ed eliminare documenti
(\emph{Delete}).

Ogni documento è raggiungibile da uno \ac{URL} univoco. Il protocollo \ac{HTTP}
prevede la seguente sintassi per definire l'~\ac{URL} di una risorsa alloccata : 

http\_URL = ``http:'' ``//'' host [ ``:'' port ] [ abs\_path [ ``?'' query
]]
\\
\emph{Host} è l'indirizzo del server sul quale è ospitato il database,
\emph{port} indica la porta sulla quale è in ascolto CouchDB, che di default è
la 5984 per richieste http, 6984 per ssl, modificabili accedendo al file di
configurazione. Se si vuole accedere al database \emph{abs\_path} è il nome del
database stesso, se si vuole accedere ad una vista il percorso da inserire sarà
composto da

nome\_database ``/\_design/'' nome\_design\_document ``/\_view/'' nome\_vista.
\\\\
Le operazioni \ac{CRUD}  sulle risorse sono
eseguibili tramite i seguenti metodi \ac{HTTP} ( gli esempi riportati sono stati
realizzati con il command line tool \textit{cURL}):
\begin{itemize}
  \item \textbf{GET} : richiede l'elemento specificato. Il formato dell'\ac{URL}
  definisce ciò che viene restituito: CouchDB può restituire elementi
  statici (ad esmepio viste), documenti di database e di configurazione.
  L'informazione viene restituita sotto forma di un documento \ac{JSON}.
  Per ottenere documenti non l'URL deve essere composto da :
  
  ``http://'' host ``:'' port ``/'' database ``/'' \_id
  
  es: \emph{curl -X GET ``http://atcustomersat:5984/padova/55090\ldots''}.
  
  Per ottenere elementi statici come le viste l'URL deve essere composto da:
  
  ``http://'' host ``:'' port  ``/'' database ``/\_design/''\_id 
  
  es: \emph{curl -X GET
  ``http://atcustomersat:5984/padova/\_design/consulting''}.
  
  \item \textbf{HEAD} : questo metodo viene utilizzato per ottenere
  l'intestazione \ac{HTTP} di una richiesta GET senza il corpo della risposta.
  Viene utilizzato per verificare l'esistenza di un oggetto senza richiederne il
  contenuto. In caso di documenti di grandi dimensioni, verificarne l'esistenza
  con il metodo HEAD velocizza notevolmente l'applicazione, in quanto non si
  deve aspettare che ritorni l'oggetto richiesto. L'\ac{URL} per accedere ai
  documenti ha la stessa sintassi di una richiesta GET.
   
  es: \emph{curl -X HEAD ``http://atcustomersat:5984/padova/55090\ldots''}
  \item \textbf{POST} : permette di inviare nuovi documenti al database di
  destinazione. In questo caso la richiesta POST deve avere come
  contenuto del body un documento \ac{JSON} valido.
  
  Per inviare un documento l'URL deve essere composto da:
  
  ``http://'' host ``:'' port ``/'' database 
  
  es : \emph{curl -X POST ``http://atcustomersat:5984/padova'' -H
  ``Content-Type:}
  \emph{application/json'' --data @scorecard.json}
  
  Dove \emph{scorecard.json} è un file contentente un documento json. 
  \item \textbf{PUT} : permette di inviare una risorsa specificata. In
  CouchDB il metodo PUT può essere utilizzato per creare nuovi dodumenti e
  database.
  
  es : \emph{curl -X PUT ``http://atcustomersat:5984/padova/123456798''}
  \emph{ -H ``Content-Type:application/json'' --data @scorecard.json}
  \item \textbf{DELETE} : cancella la risorsa specificata.
  
  es: \emph{curl -X DELETE ``http://atcustomersat:5984/padova/55090\ldots''}
\end{itemize}
Per aderire alle specifiche HTTP, è necessario, inoltre, che gli headers
siano settati e vengano forniti in modo da ottenere il formato e codifica
corretta. Gli headers essenziali sono \emph{Content~-~type} e \emph{Accept}.
\textbf{Content~-~type} specifica il tipo di contenuto dei dati che
vengono forniti nella richiesta. CouchDB accetta il formato JSON quindi l'header
sarà \emph{application/json}.
L'header \textbf{Accept}, invece, indica quali tipi di file sono accettati dal
client, in formato MIME (es: text/html). Solitamente sarà settato con
\emph{application/json}, \emph{*/*} se sono supportati tutti i tipi di file.
\\\\
Per realizzare interrogazioni parametrizzate, si usano le query-string. La
query-string è la parte di un URL che contiene dei dati da passare in input ad
un programma. Il carattere ``?'' apre la query-string, è utilizzato come
carattere speciale e non viene incluso nella query. Ogni argomento è formato da
una coppia chiave - valore. Alla chiave X si assegna il valore Y con il segno
``='' ed ogni argomento è separato da un altro con il carattere ``\&''. 
La sinstassi per realizzare una query string è, dunque, la seguente :

  
``?parametro1=valore1\&parametro2=valore2\&parametro3=valore3''.
\\\\
Le query parametrizzate vengono utilizzate nell'applicazione nel momento della
generazione dei grafici. C'è la possibilità di scegliere la sede che si vuole
monitorare, il tipo di servizio, l'intervallo temporale e il tipo di grafico.
Ogni voce scelta andrà a settare un parametro nell'\ac{URL}. I parametri
permessi da CouchDB sono illustrati nella tabella seguente.
\\\\
\begin{tabular}{|>{\columncolor{background}}p{2.4cm}|p{3.1cm}|p{1.4cm}|p{5.8cm}|}
  \hline
  \rowcolor{table}
  \textbf{Parametro} & 
  \textbf{Valore} & 
  \textbf{Default} &
  \textbf{Descrizione}
   
  \\\hline key & JSON & - & Indica la coppia chiave-valore, deve rispettare
  la sintassi dell'URL.
  \\\hline startkey & JSON
    & - & Indica il valore della chiave di partenza, deve rispettare la
    sintassi dell'URL.
  \\\hline startkey\_docid & ``\_id'' del documento & - & ID del documento di
  partenza.
  \\\hline endkey & JSON
    & - & Indica il valore massimo della chiave, deve rispettare la
    sintassi dell'URL.
  \\\hline endkey\_docid & ``\_id'' del documento & - & ID del documento che
    termina la ricerca.
  \\\hline limit & intero & - & Limita il numero di documenti
    restituiri.
  \\\hline descending & true/false & false & Ordinamento della ricerca.
  \\\hline skip & intero & 0 & Salta il numero indicato di documenti.
  \\\hline group & true/false & false & Controlla se la funzione di
    \emph{reduce} viene applicata a un gruppo di chiavi distinte o singola riga di
    risultati.
  \\\hline group\_level & intero & - & Indica il livello di raggruppamento da
  applicare alla funzione \emph{reduce}.
  \\\hline reduce & trey/false & true & Specifica se utilizzare la funzione
  \emph{reduce}.
  \\\hline
\end{tabular}
\\\\
Ad esempio per ricevere i dati di una particolare sede, ragguppati per mesi e
per il periodo temporale compreso tra gennaio e luglio 2013, la query è la
seguente

?group\_level=2\&startkey=[2013,1,1,1]\&endkey=[2013,7,31,23]