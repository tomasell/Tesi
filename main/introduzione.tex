\section{Introduzione}
A partire dagli anni novanta del XX secolo la tecnologia, lo sviluppo e la
competizione nel mercato hanno portato le grandi aziende, soprattutto operanti nel settore
terziaro del lavoro, ad avere la necessità di monitorare la qualità dei loro
servizi. 
La scelta di un approccio di mercato del tipo \textit{customer
oriented} rende necessaria l'effettuazione di indagini interne
mirate a monitorare il \textit{modus operandi} dei dipendenti,la competenza,il clima di gruppo, le
risorse e le difficoltà. Di fondametale importanza risultano anche le verifiche
proposte per analizzare la soddisfazione e i bisogni dei clienti, al fine di
proporre nuovi servizi o per migliorare quelli già offerti, così da poter
continuare ad essere non solo competitivi nel mercato, ma leader nel settore al
quale appartengono. Permettere al cliente di esprimere un'opinione sulla qualità
dei servizi offerti, da' la possibilità di mettere in evidenza quali sono le
caratteristiche che fanno eccellere la ditta e favorisce l'individuazione degli
aspetti su cui bisogna investire risorse per migliorarli.

Al fine di cogliere e valutare il livello di soddisfazione è stato idealizzato
il modello \textit{SERVQUAL} (Parasuraman, Zeithaml e Berry). Il
\textit{SERVQUAL} è costituito da due serie di 22 domande predefinite alle quali
corrispondono delle risposte con valutazione numerica con scala da 1 a 7 ,
mettendo a confronto le aspettative generiche del cliente nei confronti del
servizio e la percezione del prodotto offerto.
Il modello  consente di misurare il CS per 5 elementi fondamentali del
servizio: 
\begin{itemize}
  \item \textbf{Elementi tangibili}: aspetto delle strutture fisiche,
  delle attrezzature e del personale;
  \item \textbf{Affidabilità}: capacità di erogare il servizio promesso
  in modo affidabile e preciso;
  \item \textbf{Capacità di risposta}: volontà di aiutare i clienti e di
  fornire il servizio con prontezza;
  \item \textbf{Capacità di rassicurazione} : competenza e cortesia
  degli impiegati e loro capacità di  ispirare fiducia e sicurezza;
  \item \textbf{Empatia} : assistenza premurosa e individualizzata che
  l’azienda riserva ai suoi clienti.
\end{itemize}

I risultati del confronto fra \textit{attese} (\textbf{A}) e \textit{percezioni}
(\textbf{P}) di qualità possono essere di tre tipi:
\begin{itemize}
  \item P > A : la qualità del servizio è molto alta perchè le
  percezioni superano le aspettative;
  \item P = A : la qualità del servizio è buona perchè si sono soddisfatte
  in pieno le attese del cliente;
  \item P < A : la qualità del servizio è bassa. Se P è
  molto vicino, cioè solo di poco inferiore ad A e queste ultime erano
  alte, il risultato deve essere considerato buono perchè si sono soddisfatte
  quasi del tutto le elevate attese di qualità del cliente. Questo è tanto
  più vero quanto più il servizio è complesso tecnicamente o sofisticato come
  immagine, ecc. Il risultato sarà, invece, negativo solo se P è abbastanza o
  molto inferiore ad A, per cui il cliente è abbastanza o molto deluso dalla
  qualità del servizio.
\end{itemize}

Durante il periodo di stage che ho sostenuto presso SAIV S.p.A.
in collaborazione con il tutor aziendale Lovato Giovanni e il collega Rigoni
Giulio , è stata sviluppata un'applicazione per monitorare 



