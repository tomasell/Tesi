\newpage
\section{Promise Pattern}
Per registrare il voto espresso dall'utente si effettua una richiesta POST al
server dove è ospitato il dabatase. Le richieste vengono inoltrate con
metodologia \ac{AJAX}. \ac{AJAX} permette di
effettuare delle chiamate asincrone, cioè chiamate ad una risorsa
esterna che non interferiscono con l'esecuzione della risorsa chiamante. I
risultati della risorsa esterna saranno utilizzabili solo quando disponibili.
Chiamate di tipo asincrono, quindi, permettono di effettuare altre operazioni
senza alcun redirect o refresh della pagina.

La richiesta HTTP viene composta dal metodo \$.ajax() prendendo in
considerazione i parametri d'ingresso. Settando il metodo da chiamare, gli
header necessari e il body, la richiesta viene composta e inviata al server.
Utilizzare chiamate asicnrone, permette di strutturare il codice in nuovo modo:
mentre le richieste \ac{AJAX} vengono eseguite è possibile realizzare delle
callback che possono essere chiamate a seconda dello stato della chiamata
utilizzando il promise pattern. 

Una promessa è un oggetto che rappresenta il valore di ritorno o l'eccezione
lanciata che una funzione può eventualmente fornire. Il promise pattern prevede
tre differenti stati di una promessa :
soddisfatta ( \emph{fulfilled} ) nel caso in cui la funzione ritorni un valore,
non soddisfatta ( \emph{rejected} ) nel caso la funzioni generi un'eccezione e in
corso ( \emph{progress}).
Lo stato progess è uno stato temporaneo, mentre fulfilled e rejected sono
definitivi. Una promessa può passare da uno stato progess a uno degli
altri due, ma non viceversa. Il costrutto delle promesse permette quindi di
realizare delle funzioni che vengono chiamate a seconda dello stato
dell'oggetto. Le funzioni sono create all'interno del metodo
then(~doneCallbacks [, failCallbacks] [, progressCallbacks ]~) riferito alla
funzione iniziale.

Con l'uso delle promesse la scrittura del codice viene semplificata ed agevola
il controllo degli errori. Se pensiamo ad una serie di eventi da realizzare in
catena, una funzione viene scritta all'interno dell'altra, e richiede
particolare attenzione dal parte del programmatore per evitare gli errori. Ad
esempio il codice:
\\\\
\begin{lstlisting}[language=JavaScript] 
step1(function (value1) {
    step2(value1, function(value2) {
        step3(value2, function(value3) {
            step4(value3, function(value4) {
                // Do something with value4
            });
        });
    });
}); 
\end{lstlisting}
può essere semplificato con le promesse. Lo stato di una promessa viene
propagato, quindi possiamo cogliere il fallimento di una di esse nel
momento in cui fallisce, o alla fine della catena di promesse. In
questo caso possiamo riscrivere il codice presentato prima con il seguente:
\\\\
\begin{lstlisting}[language=JavaScript] Q.fcall(step1) .then(step2)
.then(step3)
.then(step4)
.then(function (value4) {
    // Do something with value4
}, function (error) {
    // Handle any error from step1 through step4
})
.done();
\end{lstlisting}
L'utilizzo delle promesse facilita anche l'utilizzo di chiamate parallele. Il
metodo \emph{all} riceve un array di promesse, ed appena tutte
sono risolte si scatenano gli eventi della callback. In altri casi è possibile
attendere che solo una di un array di promesse venga soddisfatta e si utilizza
allora il metodo \emph{any}. Un'ulteriore alternativa consiste nell'attendere un
numero precisato di promesse soddisfatte utilizzando la funzione \emph{some}.
\\\\
Di seguito il codice utilizzato nell'applicazione per inviare la votazione.
Il metodo ajax() riceve come parametri d'ingresso l'URL del server,   
il metodo \ac{HTTP} e l'oggetto da memorizzare nella base di dati. In questo
caso la promessa, quando e se viene soddisfatta, lancia la funzione che restituisce un
feddback all'utente della corretta esecuzione del processo di voto. 

 \begin{lstlisting}[language=JavaScript] 
$.ajax('http://' + server +':8888/' + agency, { 
    type : "POST",
    contentType : "application/json",
    processData : false,
    data : JSON.stringify({
      "agency" : agency,
      "operation" : operation,
      "score" : $(this).data('score'),
      "date" : new Date(),
      "comment" : $('#comment').val(),
      "customerName" : $('#customerName').val(),
      "email" : $('#email').val()
    })
  }).then(function(response) {
      $('#feedback').modal({
        backdrop : 'static'
      });
    });
\end{lstlisting}