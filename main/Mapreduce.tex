\newpage
\subsection{Map-Reduce}
\emph{MapReduce} è un framework software per scrivere facilmente applicazioni
che elaborano grandi quantità di dati in parallelo. Il sistema è formato da due
funzioni~: \emph{map} e \emph{reduce}. La prima elabora le coppie chiave-valore
, la seconda elabora il set di coppie generate dalla funzione \emph{map} 
restituendo l'output richiesto.
\begin{itemize}
  \item \emph{map}: il nodo master prende i dati di ingresso (fase
  \emph{input}), li suddivide in piccoli blocchifase
  \emph{splitting}, generalmente da 64MB o 128MB e distribuisce il lavoro ai
  nodi slaves. Il singolo nodo \emph{mapper} produce il risultato intermedio della
  funzione di map() sotto forma di coppie [chiave,valore] memorizzate su un file
  distribuito, la cui locazione è notificata al master alla fine della fase di
  map (fase \emph{mapping}).
  \item \emph{reduce}: il nodo master colleziona le risposte, combina le coppie
  [chiave,valore] in liste di valori che condividono la stessa chiave e li
  ordina per chiave (fase
  \emph{shuffling}). Le coppie della forma,
  [chiave,IteratorList(valore,\ldots)] sono passate ai nodi reducer che
  eseguono la funzione di reduce() (fase
  \emph{reducing}).
\end{itemize}
Le funzioni di \emph{map} e \emph{reduce} sono implementate in Javascript, e
CouchDB mette a disposizione ulteriori funzioni native da utilizzare.
La funzione \emph{reduce} come parametri di input ha una collection di chiavi,
una collection di valori e un valore booleano \emph{rereduce}. Se settato a
true, rereduce permette di usare la funzione reduce al blocco di risultati
generato dalle chiamate della funzione. Per database di grandi dimensioni,
CouchDB chiama le funzioni per blocchi di dati in modo da velocizzare l'analisi
dei dati. Non applicando la funzione rereduce c'è il rischio di ottenere un
output sbagliato.

 Ad esempio ipotizziamo di avere un set di documenti
composto da 1000 chiavi.
La fase di map calcola ``chiave''=``valore'' e il motore di CouchDB divide i
documenti ricevuti come input in 3 blocchi. Nella fase di reduce, la funzione
calcola quanti elementi sono presenti nel db. La situazione di partenza prevede
1000 oggetti, la fase di map e reduce divide in 3 blocchi contenenti
rispettivamente 333,333 e 334 oggetti.
L'output resituito dalla funzione reduce è errato, in quanto calcola il numero
di oggetti presenti restituito dopo le operazioni, calcola quindi 3
oggetti(~Figura~\ref{fig:reduce}).
Applicando la funzione di rereduce, invece, viene calcolata la somma dei valori
restituiti dai 3 oggetti temporanei, restituendo l'output
corretto(~Figura~\ref{fig:rereduce}).


\begin{figure}[!h]
\begin{tikzpicture}[%
  >=triangle 60,              % Nice arrows; your taste may be different
  start chain=going below,    % General flow is top-to-bottom
  node distance=6mm and 60mm, % Global setup of box spacing
  every join/.style={norm},   % Default linetype for connecting boxes
  ]
\tikzset{
  base/.style={draw, on chain, on grid, align=center, minimum height=4ex},
  proc/.style={base, rectangle, text width=10em},
  map/.style={base, rectangle, text width=10em},
  cong/.style={->, draw, lccong}, 
  it/.style={font={\small\itshape}}
}
\node
[proc,xshift=12em](input){\textbf{key}~,~\textbf{val}\\1~,~1\\\ldots~,\ldots\\1000~,~5984};
\node [proc,xshift=-12em](split1){\textbf{key}~,~\textbf{val}\\1~,~1\\\ldots~,
\ldots\\333,210};
\node [map,fill=background](map1){function(key,val)~\{\\ count(val);\}};
\node [proc](result1){\textbf{key} , \textbf{val}\\ null~,~333};
\node [proc,xshift=12em,yshift=15.19em](split2){\textbf{key}
,\textbf{val}\\334~, 458\\\ldots~,~\ldots\\666,131}; 
\node [map,fill=background](map2){function(key,val)~\{\\ count(val);\}}; 
\node [proc](result2){\textbf{key} , \textbf{val}\\ null~,~333}; 
\node [proc,xshift=12em,yshift=15.19em](split3){\textbf{key}
,\textbf{val}\\667~,~125\\\ldots~,\ldots\\1000,5984}; 
\node [map,fill=background](map3){function(key,val)~\{\\ count(val);\}}; 
\node [proc](result3){\textbf{key} ,\textbf{val}\\ null~,~334};
\node [map,fill=background,xshift=-12em](reduce){function(key,val)~\{\\ count(val);\}};
\node [proc](output){\textbf{key} , \textbf{val}\\null~,~3};

\path (input.south) to node [near start, xshift=5em] {}
(split1); \draw [->] (input.south) -- (split1.north);
\path (input.south) to node [near start, xshift=5em] {}
(split2); \draw [->] (input.south) -- (split2.north);
\path (input.south) to node [near start, xshift=5em] {}
(split3); \draw [->] (input.south) -- (split3.north);

\path (split1.south) to node [near start, xshift=5em] {}
(map1); \draw [->] (split1.south) -- (map1.north);
\path (split2.south) to node [near start, xshift=5em] {}
(map2); \draw [->] (split2.south) -- (map2.north);
\path (split3.south) to node [near start, xshift=5em] {}
(map3); \draw [->] (split3.south) -- (map3.north);

\path (map1.south) to node [near start, xshift=5em] {}
(result1); \draw [->] (map1.south) -- (result1.north);
\path (map2.south) to node [near start, xshift=5em] {}
(result2); \draw [->] (map2.south) -- (result2.north);
\path (map3.south) to node [near start, xshift=5em] {}
(result3); \draw [->] (map3.south) -- (result3.north);

\path (result1.south) to node [near start, xshift=5em] {}
(reduce); \draw [->] (result1.south) -- (reduce.north);
\path (result2.south) to node [near start, xshift=5em] {}
(reduce); \draw [->] (result2.south) -- (reduce.north);
\path (result3.south) to node [near start, xshift=5em] {}
(reduce); \draw [->] (result3.south) -- (reduce.north);

\path (reduce.south) to node [near start, xshift=5em] {}
(output); \draw [->] (reduce.south) -- (output.north);

\end{tikzpicture} 
\caption{MapReduce con \emph{rereduce=false}}\label{fig:reduce}
\end{figure}

\def\HS{\hspace{\fontdimen2\font}}

\begin{figure}[!h]
\begin{tikzpicture}[%
    >=triangle 60,              % Nice arrows; your taste may be different
    start chain=going below,    % General flow is top-to-bottom
    node distance=6mm and 60mm, % Global setup of box spacing
    every join/.style={norm},   % Default linetype for connecting boxes
    ]
\tikzset{
  base/.style={draw, on chain, on grid, align=center, minimum height=4ex},
  empty/.style={ on chain, on grid, align=center, minimum height=6ex,text
  width=10em},
  proc/.style={base, rectangle, text width=10em, align=center},
  map/.style={base, rectangle, text width=10em,align=left},
  cong/.style={->, draw, lccong}, 
  it/.style={font={\small\itshape}}
}
\node [empty, densely dotted, it] (empty1) { };
\node [proc, densely dotted, it,xshift=12em] (p0) { splitting \space  \&
\space mapping }; \node [map,fill=background](reduce){function(key,val)~\{\\
if (rereduce)\\ \{\space\space\space sum(val) \} \\ else\\
\{\space\space\space count(val) \};\}}; \node [proc](output){\textbf{key} ,
\textbf{val}\\null~,~1000};

\path (p0.south) to node [near start, xshift=5em] {}
(reduce); \draw [->] (p0.south) -- (reduce.north);

\path (reduce.south) to node [near start, xshift=5em] {}
(output); \draw [->] (reduce.south) -- (output.north);

\end{tikzpicture} 
\caption{MapReduce con \emph{rereduce=true}}\label{fig:rereduce}
\end{figure}



\newpage \newpage
Di seguito è presentata la funzione map utilizzata nel calcolo della somma dei
voti per l'operazione di consulenza. Nel momento della fase di map, vengono scartati
i documenti che non appartengono al gruppo scelto. Successivamente viene creata
la coppia chiave-valore da inviare alla funzione di reduce. La chiave è la data
trasformata in un array, per agevolare il raggruppamento in fase di output.

\begin{lstlisting}[language=JavaScript] 
function(doc) {
  if (doc.operation != 'consulting') 
    return;
  var date = new Date(doc.date);
  emit(
    [date.getFullYear(), 
     (date.getMonth()  +1),
     date.getDate(),
     date.getHours()],
    doc.score); 
}
\end{lstlisting}
La funzione reduce raccoglie i dati, distinguendo i casi in cui rereduce è
settato a true oppure a false.
 \begin{lstlisting}[language=JavaScript]
function (key, values, rereduce) {
  // Reduce function
  var count = 0;
  var sum = 0;
  var i;

  if(!rereduce) {
    // `values` stores actual map output
    for(i = 0; i < values.length; i++) {
      count += 1;
      sum += Number(values[i]);
    }
    return {"count":count, "sum":sum};
  }

  else {
    // `values` stores count/sum objects returned previously.
    for(i = 0; i < values.length; i++) {
      count += values[i].count;
      sum   += values[i].sum;
    }
    return {"count":count, "sum":sum};
  }
}
\end{lstlisting}