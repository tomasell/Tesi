A partire dagli anni novanta del XX secolo la tecnologia, lo sviluppo e la com-
petizione nel mercato hanno portato le grandi aziende, soprattutto operanti nel
settore terziaro del lavoro, ad avere la necessità di monitorare la qualità dei
loro servizi.

La scelta di un approccio di mercato del tipo \emph{customer oriented} rende
necessaria l'effettuazione di indagini interne mirate a monitorare il modus
operandi dei dipendenti, la competenza, il clima di gruppo, le risorse e le
difficoltà. Di fondamentale importanza risultano anche le verifiche proposte per
analizzare la soddisfazione e i bisogni dei clienti, al fine di proporre nuovi
servizi o per migliorare quelli già offerti. I questo modo l'azienda può
continuare ad essere non solo competitiva nel mercato, ma leader nel settore al
quale appartiene.
Permettere al cliente di esprimere un'opinione sulla qualità dei servizi
offerti, da' la possibilità di mettere in evidenza quali siano le
caratteristiche che fanno eccellere la ditta e favorisce l'individuazione degli
aspetti su cui bisogna investire risorse per migliorarli.
Al fine di cogliere e valutare il livello di soddisfazione è stato idealizzato
il modello \emph{SERVQUAL} da A. Zeithaml, A. Parasuraman, L. Berry. 

Il \emph{SERVQUAL} è un
modello costituito da due serie di 22 item predefiniti con possibilità di
risposta attraverso un valore numerico secondo una scala da 1 a 7. Le domande
facilitano un confronto tra le aspettative generiche del cliente nei confronti
del servizio e la percezione del prodotto offerto.
Lo strumento consente di misurare il livello di soddisfazione su cinque elementi
fondamentali del servizio: elementi tangibili  (area di valutazione delle
strutture fisiche, delle attrezzature e del personale); affidabilità (capacità
di erogare il servizio promesso in modo affidabile e preciso); capacità di
risposta (adeguatezza, volontà di aiutare i clienti e di fornire il servizio con
prontezza); capacità di rassicurazione (competenza e cortesia degli impiegati e
loro capacità di ispirare fiducia e sicurezza); empatia (assistenza premurosa e
individualizzata che l’azienda riserva ai suoi clienti).

I risultati del confronto fra attese (A) e percezioni (P) di qualità possono
essere di tre tipi:
\begin{itemize}
  \item P > A : la qualità del servizio è molto alta perchè le percezioni
  superano le aspettative;
  \item P = A : la qualità del servizio è buona perchè si sono soddisfatte in
  pieno le attese del cliente;
  \item P < A : la qualità del servizio è bassa. 
\end{itemize}
Lo strumento \emph{SERVQUAL}, insieme ad altre metodologie centrate sul cliente,
sulle sue aspettative e percezioni, hanno avuto ampia diffusione a livello
internazionale. Lo sviluppo della tecnologia, e in particolar modo del settore
informatico, hanno permesso in questi anni la rielaborazione di strumenti per
l’analisi della soddisfazione dei clienti, quali quello sopra presentato, e la
creazione di adattamenti o nuove applicazioni informatiche più rapide.
\\\\
La mia tesi verterà proprio sulla presentazione di una nuova applicazione web
realizzata per affiancare il \emph{SERVQUAL} ed essere una via intuitiva ed
immediata per dare all’Ente committente un feedback generale del suo servizio.




