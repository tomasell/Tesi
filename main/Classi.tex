\newpage
\section{Classi}
L'applicazione sviluppata è aderente al pattern architetturale
\ac{MVC}, paradigma che prevede la distinzione di tre diversi
componenti del sistema:
\begin{itemize}
  \item \textbf{Model} : il modello è l'entità informativa del sistema e
  fornisce i metodi per accedere ai dati utili all'applicazione. Nel nostro caso il
  modello è composto dal voto del cliente, la data e la sede. I dati vengono
  registrati su database CouchDB.
  \item \textbf{View} : la vista presenta l'interfaccia grafica all'utente e ne
  determina l'interazione col sistema, visualizza i dati contenuti nel model e
  si occupa dell'interazione tra utenti e sistema. In questo caso è
  stata sviluppata in linguaggio \ac{HTML} e \ac{CSS} e permette di selezionare
  il livello di soddisfazione.
  \item \textbf{Controller} : il controllore gestisce la comunicazione tra
  l'interfaccia grafica e il database, in questo caso è stato sviluppato in
  linguaggio \ac{JS}.
\end{itemize}
Il sistema di \ac{CS} proposto sarà composto da tre classi come descritto da
~Figura~\ref{fig:classes}. Con \emph{voting} si intende il processo di votazione
, con \emph{vote} il modello salvato su database.
\begin{figure}[!h]
  \centering
  \begin{tikzpicture}
    \begin{umlpackage}{Scorecard}
      \umlclass[name=voting]{Voting : HTML}{
          + molto soddisfatto : Button \\
          + soddisfatto : Button \\
          + mediamente soddisfatto : Button \\
          + non del tutto soddisfatto : Button
        }{
        }
      \umlclass[name=js,x=3,y=3]{Voting : JS}{
        }{
          + touch(event) : feedback \\
          + post(json) : response
        }
      \umlclass[name=vote,x=6]{Vote : JSON}{
          + agency : int \\
          + vote : int \\
          + date : Date
        }{
        }
      \umlassoc{voting}{js}
      \umlassoc{js}{vote}
    \end{umlpackage}
  \end{tikzpicture}
  \caption{Classi del sistema di~\emph{CS}}
  \label{fig:classes}
\end{figure}
\\\\L'utente agisce sulla vista del programma attivando una delle sue componenti
di controllo, nel nostro caso i \emph{bottoni} che permettono di registrare la
votazione. Il controllore che è in ascolto dell'evento, riceve l'input di
votazione ed esamina la vista per rilevarne le informazioni aggiuntive, come
l'associazione tra pulsante premuto e valore assegnato. Il controllore invia
tali informazioni al modello che effettua lo storage dei dati ed
attende la risposta per aggiornare il proprio stato. Successivamente richiede
alla vista di visualizzare il risultato della computazione. La vista, infine,
farà ritornare all'utente il risultato dell'operazione eseguita.
L'implementazione del \emph{controllore} in ascolto sull'evento votazione è
riportato in figura~~\ref{fig:votingcontroller}, la sequenza del controllore
del monitoraggio voti è riportato in figura~~\ref{fig:monitoringcontroller}.

\begin{figure}[!h]
  \centering
  \begin{tikzpicture}
    \begin{umlseqdiag}
      \umlactor[class=User]{Customer}
      \umlcontrol[class=Controller]{Scorecard}
      \umldatabase[class=Datasource]{CouchDB}
      \begin{umlcall}[op=touch(event),return=feedback]{Customer}{Scorecard}
        \begin{umlcall}[op=post(json),return=response]{Scorecard}{CouchDB}
        \end{umlcall}
      \end{umlcall}
    \end{umlseqdiag}
  \end{tikzpicture}
  \caption{Sequenza di espressione voto}
  \label{fig:votingcontroller}
\end{figure}

\begin{figure}[!h]
  \centering
  \begin{tikzpicture}
    \begin{umlseqdiag}
      \umlactor[class=User]{Administrator}
      \umlcontrol[class=Controller]{Scorecard}
      \umldatabase[class=Datasource]{CouchDB}
      \begin{umlcall}[op=browse(),return=graph]{Administrator}{Scorecard}
        \begin{umlcall}[op=get(),return=response]{Scorecard}{CouchDB}
        \end{umlcall}
      \end{umlcall}
    \end{umlseqdiag}
  \end{tikzpicture}
  \caption{Sequenza di monitoraggio voti}
  \label{fig:monitoringcontroller}
\end{figure}